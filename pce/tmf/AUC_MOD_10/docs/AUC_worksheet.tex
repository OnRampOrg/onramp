% Title: Area Under a Curve (AUC) Worksheet
% Author: Jason Regina
% Date: 5 May 2016
% Description: Supplemental worksheet for the Area Under a Curve module

%----------------------------------------------------------------------------------------
%	BEGIN WORKSHEET
%----------------------------------------------------------------------------------------
\clearpage
\setcounter{page}{1}

% Worksheet title
\begin{flushright}
{\LARGE\thetitle}
\\%
\vspace{15pt}
{\large\textit{Scaling Study Worksheet}}
\end{flushright}

% Student information fields
\begin{table}[H]
\newcolumntype{L}{>{\raggedright\arraybackslash}X}%
\label{table:exHeader}
\begin{tabu} to \textwidth { @{}L l @{}L@{} }
\textbf{Name} & & \textbf{Instructor} \\  \cmidrule(r){1-0}\cmidrule(r){3-0}
\textbf{Date} & & \textbf{Course}     \\  \cmidrule(r){1-0}\cmidrule(r){3-0}
\multicolumn{3}{@{}l}{\textbf{Type of Scaling Study}} \\ \hline
\end{tabu}
\end{table}

% PCE information fields
\begin{table}[H]
\newcolumntype{R}{>{\raggedleft\arraybackslash}X}%
\label{PCEinfo}
\begin{tabu} to 0.6\textwidth { | X[lm] | R | }
\multicolumn{2}{@{}l}{Parallel Computing Environment (PCE) Information} \\ \hline
\textbf{Hostname}        &                                              \\ \hline
\textbf{Available Nodes} &                                              \\ \hline
\textbf{Cores per Node}  &                                              \\ \hline
\end{tabu}
\end{table}

% Worksheet table
\begin{table}[H]
\centering
\label{worksheetTable}
\begin{tabu} to \textwidth { | X[cm] | X[cm] | X[cm] | X[cm] | X[cm] | }
\hline
\textbf{Processes} & \textbf{Threads per Process} & \textbf{Rectangles} & \textbf{Time} & \textbf{Efficiency} \\ \hline
                   &                              &                     &               &                     \\ \hline
                   &                              &                     &               &                     \\ \hline
                   &                              &                     &               &                     \\ \hline
                   &                              &                     &               &                     \\ \hline
                   &                              &                     &               &                     \\ \hline
                   &                              &                     &               &                     \\ \hline
                   &                              &                     &               &                     \\ \hline
                   &                              &                     &               &                     \\ \hline
                   &                              &                     &               &                     \\ \hline
                   &                              &                     &               &                     \\ \hline
                   &                              &                     &               &                     \\ \hline
                   &                              &                     &               &                     \\ \hline
\end{tabu}
\end{table}

% Instructions
\bigskip
\paragraph{Instructions}
\begin{enumerate}
\item Log on to the OnRamp web-server
\item Select a workspace
\item Select a PCE
\item Select the AUC module
\item The right-hand column summarizes important concepts from the module documentation. At this point, decide whether to conduct a \emph{strong scaling} or a \emph{weak scaling} study.
\item Input a `job\_name.' This will be used by the PCE to track your job status.
\item Input parameters into the relevant fields that are appropriate to the PCE and type of scaling studying being conducted. The AUC program accepts 5 parameters:
\begin{enumerate}
\item \textbf{onramp np:} Number of MPI processes
\item \textbf{onramp nodes:} Number of compute nodes
\item \textbf{AUC threads:} Number of OpenMP threads per process
\item \textbf{AUC rectangles:} Number of rectangles for the Riemann Sum
\item \textbf{AUC mode:} Character indicating version of AUC to run
\begin{enumerate}
\item `s' = serial
\item `o' = OpenMP only
\item `m' = MPI only
\item `h' = hybrid OpenMP and MPI
\end{enumerate}
\end{enumerate}
\item Click the `Launch Job' button
\item Find your `Run Name' among the list of `My Jobs.' Click `View Details.'
\item Record the details of the job output including the input parameters and the time.
\item Repeat the process for different parameters, depending on the scaling study being conducted.
\end{enumerate}

% Questions
\pagebreak
\paragraph{Questions}
\begin{enumerate}
\setlength\itemsep{8em}
\item Did you identify a ``sweet spot''? Where did it occur?
\item How did the run time vary with the workload?
\item Did the program scale linearly?
\item In general, how did the number of processes affect the run times? Number of threads?
\item Would you expect performance to change on another cluster? Why or why not?
\end{enumerate}
